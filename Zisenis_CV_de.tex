%-------------------------
% Resume in Latex
% Author : Jake Gutierrez
% Based off of: https://github.com/sb2nov/resume
% License : MIT
%------------------------

\documentclass[letterpaper,11pt]{article}
\usepackage[german]{babel}
% Common commands and formatting
\usepackage{latexsym}
\usepackage[empty]{fullpage}
\usepackage{titlesec}
\usepackage{marvosym}
\usepackage[usenames,dvipsnames]{color}
\usepackage{verbatim}
\usepackage{enumitem}
\usepackage[hidelinks]{hyperref}
\usepackage{fancyhdr}
\usepackage{tabularx}
\input{glyphtounicode}

% Formatting commands
\pagestyle{fancy}
\fancyhf{} % clear all header and footer fields
\fancyfoot{}
\renewcommand{\headrulewidth}{0pt}
\renewcommand{\footrulewidth}{0pt}

% Adjust margins
\addtolength{\oddsidemargin}{-0.5in}
\addtolength{\evensidemargin}{-0.5in}
\addtolength{\textwidth}{1in}
\addtolength{\topmargin}{-.5in}
\addtolength{\textheight}{1.0in}

\urlstyle{same}

\raggedbottom
\raggedright
\setlength{\tabcolsep}{0in}

% Sections formatting
\titleformat{\section}{
  \vspace{-4pt}\scshape\raggedright\large
}{}{0em}{}[\color{black}\titlerule \vspace{-5pt}]

% Ensure that generate pdf is machine readable/ATS parsable
\pdfgentounicode=1

% Custom commands
\newcommand{\resumeItem}[1]{
  \item\small{
    {#1 \vspace{-2pt}}
  }
}

\newcommand{\resumeSubheading}[4]{
  \vspace{-2pt}\item
    \begin{tabular*}{0.97\textwidth}[t]{l@{\extracolsep{\fill}}r}
      \textbf{#1} & #2 \\
      \textit{\small#3} & \textit{\small #4} \\
    \end{tabular*}\vspace{-7pt}
}

\newcommand{\resumeProjectHeading}[2]{
    \item
    \begin{tabular*}{0.97\textwidth}{l@{\extracolsep{\fill}}r}
      \small#1 & #2 \\
    \end{tabular*}\vspace{-7pt}
}

\newcommand{\resumeSubItem}[1]{\resumeItem{#1}\vspace{-4pt}}

\renewcommand\labelitemii{$\vcenter{\hbox{\tiny$\bullet$}}$}

\newcommand{\resumeSubHeadingListStart}{\begin{itemize}[leftmargin=0.15in, label={}]}
\newcommand{\resumeSubHeadingListEnd}{\end{itemize}}
\newcommand{\resumeItemListStart}{\begin{itemize}}
\newcommand{\resumeItemListEnd}{\end{itemize}\vspace{-5pt}}

% Common personal information
\newcommand{\cvname}{Jan Henrik Zisenis}
\newcommand{\cvemail}{jan@zisenis.digital}
\newcommand{\cvphone}{(+49) 15566401005}
\newcommand{\cvlocation}{Deutschland, Celle}
\newcommand{\cvwebsite}{zisenis.digital}

% Common skills
\newcommand{\cvtechskills}{Python, JavaScript (Next.js, Node.js), Postgres, Flutter, C++, Terraform}
\newcommand{\cvdevpractices}{Authentifizierung (OAuth, Magic Link), Infrastructure as Code, Embedded Systems} 

\begin{document}

\begin{center}
    \textbf{\Huge \scshape \cvname} \\ \vspace{1pt}
    \small Full-Stack Entwickler | Spezialisiert auf Next.js, Postgres, Python und Mobile Entwicklung (Flutter) \\ \vspace{1pt}
    \small \href{mailto:\cvemail}{\underline{\cvemail}} $|$ 
    \href{tel:\cvphone}{\cvphone} $|$ 
    \cvlocation $|$
    \href{https://\cvwebsite}{\underline{\cvwebsite}}
\end{center}

%-----------BERUFSERFAHRUNG-----------
\section{Berufserfahrung}
  \resumeSubHeadingListStart
    \resumeSubheading
      {Persönliche Weiterentwicklung (Karrierepause)}{Sep 2024 -- Heute}
      {Full-Stack Entwickler (Freiberufliche Projekte)}{Remote}
      \resumeItemListStart
        \resumeItem{Erstellung von Wireframes und Durchführung von 15 Usability-Tests zur Verbesserung der UX und Reduzierung von Designfehlern}
        \resumeItem{Aufbau der Infrastruktur mit Terraform für skalierbare, dekonstruierbare Deployments; Auswahl von Hetzner als Provider}
        \resumeItem{Entwicklung von zwei Next.js-Projekten mit Full-Stack-Entwicklung inklusive Postgres-Backend und OAuth/Magic Link Authentifizierung}
        \resumeItem{Backend-spezifische Arbeit: Optimierung des Datenbankschemas, Verbesserung der Abfrageeffizienz um 20\%}
        \resumeItem{Präsentation zur Förderung der Nutzung von Nix zur Optimierung von Entwicklungsumgebungen, Verbesserung der Einarbeitungszeit}
      \resumeItemListEnd

    \resumeSubheading
      {Netrocks}{Feb 2024 -- Aug 2024}
      {Softwareentwickler}{Remote}
      \resumeItemListStart
        \resumeItem{Absolvierung eines 10-wöchigen Praktikums zur Entwicklung einer White-Label Frontend-Lösung mit Flutter und Backend-API-Integration}
        \resumeItem{Verfassung der Bachelorarbeit: Konzeption und Implementierung eines wiederverwendbaren Frameworks für mobile Anwendungen, Verbesserung der Entwicklungsgeschwindigkeit um 70 Stunden/Projekt}
      \resumeItemListEnd

    \resumeSubheading
      {E3DC HagerEnergy Group}{Jan 2022 -- Dez 2022}
      {Softwareentwickler}{Remote}
      \resumeItemListStart
        \resumeItem{Automatisierung von SVN-Datei- und Ordnerübertragungen durch Entwicklung eines Python-Skripts, Reduzierung von Fehlern und manuellen Eingriffen um 70\%}
        \resumeItem{Entwicklung eines Testpakets für ein eingebettetes System, Sicherstellung 100\%iger Übereinstimmung mit Hardware-IO-Spezifikationen}
        \resumeItem{Debugging und Optimierung von Echtgeräte-IO-Operationen zur Verbesserung der Entwicklereffizienz}
      \resumeItemListEnd
  \resumeSubHeadingListEnd

%-----------PROJEKTE-----------
\section{Projekte}
    \resumeSubHeadingListStart
      \resumeProjectHeading
          {\textbf{Modulare Self-Service App für Kaffeemaschinen} $|$ \emph{Flutter, Dart, MVC, Provider}}{Feb 2024 -- Aug 2024}
          \resumeItemListStart
            \resumeItem{Entwicklung einer modularen Self-Service App für Kaffeemaschinen mit White-Label-Funktionalitäten und Material 3 Design}
            \resumeItem{Bereitstellung einer plattformübergreifenden Lösung mit Flutter und Dart}
            \resumeItem{Evaluierung von Anwendungsfällen zur Verbesserung der App-Benutzerfreundlichkeit}
          \resumeItemListEnd
      
      \resumeProjectHeading
          {\textbf{Lademanagement Backend-Entwicklung} $|$ \emph{Java, JSF, Docker, MySQL}}{Okt 2022 -- Feb 2023}
          \resumeItemListStart
            \resumeItem{Erstellung und Konfiguration von Docker-Images und Docker Compose für Tomcat mit JSF und Quarkus}
            \resumeItem{Entwicklung eines robusten Backends zur Verbindung mit einer MySQL-Datenbank für E-Fahrzeug-Ladesysteme}
            \resumeItem{Implementierung eines kollaborativen GitLab-basierten Workflows zur Verbesserung der Teamproduktivität}
          \resumeItemListEnd

      \resumeProjectHeading
          {\textbf{Embedded Systems Firmware-Entwicklung} $|$ \emph{C++, STM32}}{Mär 2023 -- Aug 2023}
          \resumeItemListStart
            \resumeItem{Programmierung von Firmware für einen Embedded-Chip mit STM32CubeIDE zur Peripherie-Kommunikation}
            \resumeItem{Entwicklung synchronisierter, race-condition-freier Kommunikationsprozesse mittels Multithreading}
            \resumeItem{Konfiguration und Implementierung der I2C-Kommunikation mit einem Beschleunigungssensor}
          \resumeItemListEnd

      \resumeProjectHeading
          {\textbf{Online Secondhand-Markt UI/UX Design} $|$ \emph{Figma, UX Design}}{Okt 2022 -- Feb 2023}
          \resumeItemListStart
            \resumeItem{Durchführung eines vollständigen Designprozesses für eine Online-Secondhand-Plattform}
            \resumeItem{Erstellung von Wireframes und Entwicklung von High-Fidelity-Prototypen mit Figma}
            \resumeItem{Durchführung von Benutzertest-Interviews und Integration von Feedback zur Verbesserung der Plattform}
          \resumeItemListEnd
    \resumeSubHeadingListEnd

%-----------FÄHIGKEITEN-----------
\section{Fähigkeiten}
 \begin{itemize}[leftmargin=0.15in, label={}]
    \small{\item{
     \textbf{Programmiersprachen \& Tools}{: Python, JavaScript (Next.js, Node.js), Postgres, Flutter, C++, Terraform} \\
     \textbf{Entwicklungspraktiken}{: Authentifizierung (OAuth, Magic Link), Infrastructure as Code, Embedded Systems} \\
     \textbf{Soft Skills}{: Teamarbeit, Präsentationsfähigkeiten und Problemlösung}
    }}
 \end{itemize}

%-----------AUSBILDUNG-----------
\section{Ausbildung}
  \resumeSubHeadingListStart
    \resumeSubheading
      {Hochschule Osnabrück}{2020 -- Aug 2024}
      {Bachelor of Engineering, Informatik}{Deutschland}
      \resumeItemListStart
        \resumeItem{Engagement als Mentor vom 3. bis 6. Semester zur Unterstützung von Erstsemester-Studierenden}
      \resumeItemListEnd
  \resumeSubHeadingListEnd

\end{document} 